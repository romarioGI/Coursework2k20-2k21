\section*{Заключение}
	\addcontentsline{toc}{section}{Заключение}

	Таким образом, цель данной работы достигнута, все поставленные задачи решены. В работе описан языка элементарной алгебры, приведены примеры задач элементарной алгебры, достаточно подробно описан алгоритм элиминации кванторов в этом языка~--- алгоритм Тарского. По мимо этого, алгоритм был реализован в виде компьютерной программы для достаточно важного случая~--- для формул без параметров. Эта программа включает в себя систему ввода формул языка элементарной алгебры, библиотеки классов для представления объектов этого языка и собственно реализацию алгоритма Тарского.

	В заключении хочется отметить, что алгоритм Тарского далеко не самый эффективный алгоритм, но он был первым в своем роде. Именно сконструировав этот алгоритм Альфред Тарский доказал, что элементарная алгебра допускает элиминацию кванторов, хотя до этого многие годы это считалось невозможным. 
	
	Для систем компьютерного доказательства, которые в ближайшем будущем станут очень востребованными (формальная верификация компьютерных программ), алгоритм Тарского вряд ли применим из-за крайне высокой трудоемкости. Поэтому можно продолжать работу в данному направлении и изучать другие алгоритмы элиминации квантров. А с точки зрения математики, интересен вопрос, а какие ещё языки допускают элиминацию кванторов? Поэтому автор продолжит работу в данном направлении.