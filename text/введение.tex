\section*{Введение}
	\addcontentsline{toc}{section}{Введение}

Пусть $\mathcal{A}$~--- формула логики высказываний. Задача: определить, является ли формула $\mathcal{A}$ тождественно истинной. В некотором смысле это <<трудная>> задача, однако очень просто предложить алгоритм для ее решения, но который будет <<не эффективным>>~--- алгоритм Британского музея. Рассмотрим другую задачу: определить, является ли формула \textbf{логики предикатов} тождественно истинной. Для данной задачи алгоритм перебора в общем случае уже не применим, так как множество значений переменных не обязано быть конечным. Но оказывается, для некоторых языков логики предикатов существуют алгоритмы решающие эту задачу. Одним из таких алгоритмов и является алгоритм Тарского, описанию и реализации которого посвящена данная работа.

\textbf{Цели работы:} изучить и описать алгоритм Тарского, и реализовать его в виде компьютерной программы.

\textbf{Задачи:}
\begin{itemize}
	\item Ввести определения, сформулировать и доказать утверждения необходимые для описания алгоритма Тарского;
	\item Реализовать компьютерную программу, которая по формуле элементарной алгебры без параметров   введенной с клавиатуры, строит эквивалентную бескванторную формулу того же языка. 
\end{itemize}

В первой части данной работы будет определен язык элементарной алгебры, дано определение элиминации кванторов и сформулировано утверждения о ней. Далее пойдет речь об идеях, на которых основан алгоритм, будут определены таблицы Тарского. Затем будут даны определения полунасыщенной и насыщенной систем многочленов, после чего будет описан метод построения таблиц Тарского, что практически завершит описание алгоритма Тарского.

Во второй части подробно рассматривается программа, написанная и отлаженная автором работы, а именно описано как происходит распознавание формулы, какие при этом используются алгоритмы, как организованно представление формул, описываются реализации насыщения системы многочленов и построения таблицы Тарского. 



